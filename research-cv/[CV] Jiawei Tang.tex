\documentclass[10pt,a4paper,sans]{./moderncv/moderncv}

\moderncvstyle{classic}
\moderncvcolor{black}

\usepackage[utf8]{inputenc}
\usepackage[scale=0.75]{geometry}
\usepackage{latexsym}
\usepackage{amsfonts}
\usepackage{amsmath}
\usepackage{amssymb}
\usepackage{txfonts}
\recomputelengths


%\name{{\huge Jiawei}}{{\huge Tang}}
\name{\huge{Jiawei}}{\huge{Tang (Research Portfolio)}}
\begin{document}
\maketitle

\vspace{-1em}

% \section{Education}
% \cvline{High School}{\textbf{American School of Doha}, Qatar.}
% \cvline{}{GPA: 4.2 \hspace{2ex} SAT: 1570 (Math: 800; EBRW: 770) \hspace{2ex} TOEFL: 120}
% \cvline{Pre-college Scholars}{\textbf{University of California, Berkeley,} USA.
% 	\newline	
% 	$\bullet$ Data 8: Foundations of Data Science. Units: 4.0. Year: 2021. Grade: \textbf{A}{\Large$\mathbf{+}$}.}

\section{Research Internships}

\cvline
{\textbf{\Large ---}}
{\large\textbf{Massachusetts Institute of Technology (MIT)}, USA (2022/06-08)}

\cvline{\textbf{Mentor}}
{Professor Samuel Madden}

\cvline{\textbf{Project}}
{Use deep learning models to solve the problem of entity resolution. Entity resolution is the task of deciding whether two data records refer to the same real-world object. It has diversified application domains such as banking, insurance, e-commerce, health care, and many others. For example, an e-commerce company wants to know if two products from different suppliers are the same so they can be displayed on the same product page; two banks sharing data need to identify and reconcile common customers.}

\cvline{\textbf{Contributions}}
{I was responsible for system design, implementation, and testing for two tasks: determining the accuracy of foundation models for entity resolution and designing a deep-learning model for generic entity resolution.}

\cvline{\textbf{Publication}}
{\textbf{First author} of paper ``Generic Entity Resolution Models'' accepted by Table Representation Learning Workshop @ NeurIPS 2022, where NeurIPS is one of the most prestigious and competitive international conferences in machine learning and computational neuroscience.}

\cvline{\textbf{Link}}
{\textcolor{blue}{\url{https://openreview.net/pdf?id=tRkVo1jMas}}}

\vspace{2ex}
\cvline
{\textbf{\Large ---}}
{\large\textbf{Qatar Computing Research Institute}, Qatar (2021/06-08)}

\cvline{\textbf{Mentor}}
{Dr. Mourad Ouzzani}

\cvline{\textbf{Project}}
{Build an end-to-end data visualization system that acts as a virtual assistant to allow novices to create visualizations through either natural language or speech.}

\cvline{\textbf{Contributions}}
{Designed and implemented two main components: Speech-to-Text which is based on Google Cloud Speech-to-Text Rest API, and Text-to-VIS, which uses an end-to-end neural machine translation model.}

\cvline{\textbf{Publication}}
{\textbf{First author} of paper ``Sevi: Speech-to-Visualization through Neural Machine Translation'' accepted by ACM SIGMOD International Conference on Management of Data, where SIGMOD is a leading international forum for database researchers. I presented and demonstrated this work in SIGMOD 2022 @Philadelphia.}

\cvline{\textbf{Link}}
{\textcolor{blue}{\url{https://dl.acm.org/doi/pdf/10.1145/3514221.3520150}}}

\vspace{2ex}
\cvline
{\textbf{\Large ---}}
{\large\textbf{Tsinghua University}, China (2020/06-08)}


\cvline{\textbf{Mentor}}
{Professor Guoliang Li}

\cvline{\textbf{Project}}
{Construct a benchmark of (natural language, data visualization) pairs and use this benchmark to train a deep learning model that translate a natural language query into a data visualization.}

\cvline{\textbf{Contributions}}
{Used Python toolkits to clean and annotate data. Used PyTorch and Transformer models to train a deep learning model to support the translation from natural language queries to data visualizations.}

\cvline{\textbf{Publication}}
{\textbf{Co-author} of paper ``Natural Language to Visualization by Neural Machine Translation'' accepted by IEEE Transactions on Visualization and Computer Graphics 2021, a top journal for data visualization.}

\cvline{\textbf{Link}}
{\textcolor{blue}{\url{https://ieeexplore.ieee.org/document/9617561}}}

\vspace{2ex}

\textcolor{blue}{Note: Please see the following three pages for the abstracts of the above three papers.}

\end{document}


